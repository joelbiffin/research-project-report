\documentclass[11pt]{report}


% To set margin width, text height, space for footnotes and all sorts
% of other settings related to the geometry of the pages in your
% report, use the 'geometry' package. 
%
% http://tug.ctan.org/cgi-bin/ctanPackageInformation.py?id=geometry

\usepackage[margin=2cm]{geometry}


% For inclusion of figures, use the 'graphicx' package. This allows
% you to later include a figure using '\includegraphics{filename}' and
% it will also for simple sizing of the figure in your text.
%
% http://tug.ctan.org/cgi-bin/ctanPackageInformation.py?id=graphicx

\usepackage{graphicx}


% You may want to add hyperlinks to your document, which can be tricky
% to typeset because they often have weird symbols in them which LaTeX
% interprets in its own way. Use 'hyperref' for this. It also has many
% options to add e.g. a list of thumbnails to your pdf file.
%
% http://tug.ctan.org/cgi-bin/ctanPackageInformation.py?id=hyperref

\usepackage{hyperref}



% Finally, for various useful tools for mathematics typesetting, use
% the 'amsmath' package with a few assorted companion packages which
% provide extra symbols.
%
% http://tug.ctan.org/cgi-bin/ctanPackageInformation.py?id=amsmath
% http://tug.ctan.org/cgi-bin/ctanPackageInformation.py?id=amssymb

\usepackage{amsmath}
\usepackage{amssymb}

% For figure/table caption settings.

%\usepackage{varwidth}
%\usepackage[noonelinecaption=true]{caption}



\begin{document}

% The 'article' document class provides a simple way to make a title
% page: 

\title{A report}
\author{Someone}
\maketitle

% A table of contents can be generated automatically as well:

\tableofcontents


% Now comes the true content. 

\chapter{Background}
  \section{A section}
    

    Some text here. You can \emph{emphasise it} or {\bfseries make it
      bold-face} or {\large larger} or \emph{\bfseries\large combine those
      settings}. A random equation:
    \begin{equation}
    \label{e:Einstein1}
    E=mc^2 \,.
    \end{equation}
    
    If you want to align equations, use either {\tt align} or {\tt
      aligned}. See the examples below. The difference is the way in which
    the equation numbers are added. Here is the option with the two
    equations receiving only one number:
    \begin{equation}
    \label{e:Einstein2}
    \begin{aligned}
    E &= mc^2  \\
    i\hbar \frac{\partial}{\partial t} |\psi, t\rangle 
      &= \hat{H} |\psi,t\rangle\,,
    \end{aligned}
    \end{equation}
    and here is the one where every equation gets its own number:
    \begin{align}
    \label{e:Einstein3}
    E &= mc^2  \\
    \label{e:Schroedinger}
    i\hbar \frac{\partial}{\partial t} |\psi, t\rangle 
      &= \hat{H} |\psi,t\rangle\,,
    \end{align}




\chapter{Another chapter}

  \section{Again a section}

    You can cite any paper or book which is in your bibliography list
    (see below). Papers everyone should read are~\cite{Einstein:1935rr}
    and \cite{Feynman:1948ur}.

    You can refer to equations, such as~\eqref{e:Einstein3}, or to
    chapters, such as~\ref{c:intro}, or figures, such as figure~\ref{f:lion}.

    \begin{figure}[h]
    \begin{center}

    \end{center}
    \caption{This is the lion which is the \TeX{} mascotte.\label{f:lion}}
    \end{figure}


    And here is a link to a web page: \url{http://www.google.com}.






\begin{thebibliography}{50}
  \bibitem{ascher}
    U.M.~Ascher, L.R.~Petzold,
    ``Computer Methods for Ordinary Differential Equations and Differential-Algebraic Equations''
    Philadelphia: SIAM  (1935).


\end{thebibliography}

\end{document}