\documentclass[11pt,a4paper]{article}


% To set margin width, text height, space for footnotes and all sorts
% of other settings related to the geometry of the pages in your
% report, use the 'geometry' package. 
%
% http://tug.ctan.org/cgi-bin/ctanPackageInformation.py?id=geometry

\usepackage[margin=2cm]{geometry}


% For inclusion of figures, use the 'graphicx' package. This allows
% you to later include a figure using '\includegraphics{filename}' and
% it will also for simple sizing of the figure in your text.
%
% http://tug.ctan.org/cgi-bin/ctanPackageInformation.py?id=graphicx

\usepackage{graphicx}


% You may want to add hyperlinks to your document, which can be tricky
% to typeset because they often have weird symbols in them which LaTeX
% interprets in its own way. Use 'hyperref' for this. It also has many
% options to add e.g. a list of thumbnails to your pdf file.
%
% http://tug.ctan.org/cgi-bin/ctanPackageInformation.py?id=hyperref

\usepackage{hyperref}



% Finally, for various useful tools for mathematics typesetting, use
% the 'amsmath' package with a few assorted companion packages which
% provide extra symbols.
%
% http://tug.ctan.org/cgi-bin/ctanPackageInformation.py?id=amsmath
% http://tug.ctan.org/cgi-bin/ctanPackageInformation.py?id=amssymb

\usepackage{amsmath}
\usepackage{amssymb}

% For figure/table caption settings.

%\usepackage{varwidth}
%\usepackage[noonelinecaption=true]{caption}

\usepackage{enumitem}


\begin{document}

% The 'article' document class provides a simple way to make a title
% page: 

\title{Report Plan}
\author{Joel Biffin}
\maketitle

% A table of contents can be generated automatically as well:

% \tableofcontents


% Now comes the true content. 

\section{Background}

\begin{itemize}
  \item{Definition of IVPs (standard 1st order form).}
  \item{Where they come from (modelling natural phenomena).}
  \item{Reducing nth order IVPs to standard (vector) form.}
  \item{How evolutionary problems modelled by PDEs can be reduced to discretised IVPs.}
\end{itemize}

\section{Numerical Methods}

\begin{itemize}
  \item{One-step methods, explicit vs implicit.}
  \item{Local truncation error and the order of a method.}
  \item{Stability: region of stability, A-stability and L-stability.}
  \item{Runge-Kutta methods.}
  \item{Linear multi-step methods: Adams’, BDF, Nystrom.}
  \item{Predictor corrector methods and pairings.}
  \item{Variable time-mesh.}
\end{itemize}

\section{Software Design}

\begin{itemize}
  \item{Choice of python and numpy.}
  \item{Why choose to use object-oriented framework (future considerations).}
  \item{Briefly discuss design using UML.}
  \item{Difficulties that this presented in development.}
  \item{Graphing results (very brief).}
\end{itemize}

\section{Analysis of Numerical Methods}

\begin{itemize}
  \item Use scalar examples to demonstrate stability, stiffness and natural characteristics of each method.
     \begin{itemize}[label=$\star$]
        \item Non-stiff example, showing Forward Euler method vs Adams-Bashforth method with same step-size – graph Local truncation errors too.
        \item Stiff example, showing Forward Euler vs Backward Euler with multiple step-sizes (emphasising region of stability).
        \item Stiff example, similarly for Adams-Bashforth vs Adams-Moulton.
        \item Stiff example, show Adams-Moulton vs BDF2 – graph local truncation errors.
     \end{itemize}
  \item Showing comparison of predictor-corrector pairings with fixed mesh – graph local truncation errors.
    \begin{itemize}[label=$\star$]
      \item Plot graph of step-sizes against local-truncation errors.
    \end{itemize}
  \item Showing comparison of predictor-corrector pairings with variable mesh.
    \begin{itemize}[label=$\star$]
      \item Demonstrate how milne’s device is used with set tol and its impact on accuracy and stability of methods.
    \end{itemize}
\end{itemize}

\section{Case Studies}

\begin{itemize}
  \item Single pendulum (no damping-simple harmonic motion).
    \begin{itemize}[label=$\star$]
      \item Demonstrate natural damping feature of fixed step Forward-Backward Euler predictor-corrector, vs Adams-Bashforth-Moulton-2 predictor corrector (no damping).
    \end{itemize}
  \item Damped single pendulum.
  \item Damped forced pendulum.
  \item Double pendulum (chaotic motion).
\end{itemize} 


\section{Future Development}

\begin{itemize}
  \item Choice of step-size update function and local truncation error estimate in predictor-corrector (limitations to Milne’s device).
  \item Refactoring code to use Generalised LMMs and k-step BDF methods with variable coefficients so that the order of methods can be updated on-the-fly dependent on local truncation error estimates.
  \item Refactoring code to not store the entirety of results in memory – automated garbage collection or saving to disc. 
\end{itemize}


\end{document}
